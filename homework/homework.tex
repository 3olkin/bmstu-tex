%!TEX TS-program = xelatex
\documentclass[a4paper,14pt]{article}
%%% Работа с русским языком
\usepackage[english,russian]{babel} %% загружает пакет многоязыковой вeрстки
\usepackage{fontspec} %% подготавливает загрузку шрифтов Open Type, True Type и др.
\defaultfontfeatures{Ligatures={TeX},Renderer=Basic} %% свойства шрифтов по умолчанию
\setmainfont[Ligatures={TeX,Historic}]{Times New Roman} %% задаeт основной шрифт документа
\usepackage{indentfirst}
\frenchspacing

\usepackage{mathtext}
\usepackage{mathtools}
\mathtoolsset{showonlyrefs=true}
\usepackage{amsmath,amssymb} % AMS
\usepackage{icomma}

%%% Other packages
\usepackage{extsizes} %% расширяет диапазон доступных кеглей
\usepackage{setspace}
\usepackage{enumitem}
\usepackage{pdfpages}
\usepackage{fancyvrb}
\usepackage{fvextra}
\usepackage{icomma}
\usepackage{geometry}
\geometry{top=20mm}
\geometry{bottom=20mm}
\geometry{left=30mm}
\geometry{right=15mm}

%% переопределяем символ для уровней нумерованного списка
\renewcommand{\labelenumi}{\arabic{enumi})}
%% переопределяем формат номеров секций и сабсекций
\renewcommand\thesection{\arabic{section}.}
\renewcommand\thesubsection{\thesection\arabic{subsection}.}

%%% Пример
\newcounter{example}[section]
\setcounter{example}{1}
\newenvironment{example}{\refstepcounter{example}\par\medskip\textbf{Пример \thesection\theexample.}~}{\medskip}
%%% Упражнение
\newenvironment{exercise}{\par\medskip\textbf{Упражнение.}~}{\medskip}
%%% Определение
\newcounter{definition}[section]
\setcounter{definition}{5}
\newenvironment{definition}{\refstepcounter{definition}\par\medskip\textbf{Определение \thesection\thedefinition.}~}{\medskip}

\begin{document}
\onehalfspacing
%% Заглушка для секции, которая должна быть где-то раньше в книге
\section*{}
\setcounter{section}{1}
\noindent и достаточно $B$>>), которое истинно, когда высказывания $A$ и $B$ истинны или ложны одновременно.
\par Наряду с $A \Leftrightarrow B$ в эквиваленции используется запись: $A = B$.
\par Итак, при помощи логических операций построено множество высказываний, которое называют \emph{алгеброй высказываний}.
\par Основные формулы алгебры высказываний следующие:
\begin{enumerate}[ref=\arabic{enumi}]
    \item $\neg (\neg A) = A$ \label{enum:1:1};
    \item законы \emph{дистрибутивности} (распределительные законы): \\
          $A \wedge (B \vee C) = (A \wedge B) \vee (A \wedge C)$; \\
          $A \vee (B \wedge C) = (A \vee B) \wedge (A \vee C)$;
    \item законы де Моргана\footnote[2]{\small{~\emph{Огастес де Морган} (1806-1871) ~--~ шотландский математик, логик, основоположник логической теории отношений}}: \\
          $\neg(A \wedge B) = (\neg A) \vee (\neg B)$; \\
          $\neq(A \vee B) = (\neg A) \wedge (\neg B)$;
    \item $A \Rightarrow B = (\neg A) \vee B$ \label{enum:1:4};
    \item $\neg(A \Rightarrow B) = A \wedge (\neg B)$ \label{enum:1:5}.
\end{enumerate}
\par Эти формулы могут быть доказаны сравнением соответствующих таблиц истинности.
\begin{example}
    Докажем формулу \ref{enum:1:5}). Таблица истинности для утверждения в левой части имеет вид:
    \begin{center}
        \begin{tabular}{|c|c cc ccc|}
            \hline
            ~$A$~      & ~$B$~      & \multicolumn{2}{|c|}{~$A \Rightarrow B$~} & \multicolumn{3}{|c|}{~$\neg(A \Rightarrow B)$~} \\
            \hline
            ~\emph{и}~ & ~\emph{и}~ & \multicolumn{2}{|c|}{~\emph{и}~}          & \multicolumn{3}{|c|}{~\emph{л}~}                \\
            \hline
            ~\emph{и}~ & ~\emph{л}~ & \multicolumn{2}{|c|}{~\emph{л}~}          & \multicolumn{3}{|c|}{~\emph{и}~}                \\
            \hline
            ~\emph{л}~ & ~\emph{и}~ & \multicolumn{2}{|c|}{~\emph{и}~}          & \multicolumn{3}{|c|}{~\emph{л}~}                \\
            \hline
            ~\emph{л}~ & ~\emph{л}~ & \multicolumn{2}{|c|}{~\emph{и}~}          & \multicolumn{3}{|c|}{~\emph{л}~}                \\
            \hline
        \end{tabular}
    \end{center}
\end{example}

\newpage
\noindent А таблица истинности для утверждения в правой части имеет вид:
\begin{center}
    \begin{tabular}{|c|c cc ccc|}
        \hline
        ~$A$~      & ~$B$~      & \multicolumn{2}{|c|}{~$\neg B$~} & \multicolumn{3}{|c|}{~$A \wedge (\neg B)$~} \\
        \hline
        ~\emph{и}~ & ~\emph{и}~ & \multicolumn{2}{|c|}{~\emph{л}~} & \multicolumn{3}{|c|}{~\emph{л}~}            \\
        \hline
        ~\emph{и}~ & ~\emph{л}~ & \multicolumn{2}{|c|}{~\emph{и}~} & \multicolumn{3}{|c|}{~\emph{и}~}            \\
        \hline
        ~\emph{л}~ & ~\emph{и}~ & \multicolumn{2}{|c|}{~\emph{л}~} & \multicolumn{3}{|c|}{~\emph{л}~}            \\
        \hline
        ~\emph{л}~ & ~\emph{л}~ & \multicolumn{2}{|c|}{~\emph{и}~} & \multicolumn{3}{|c|}{~\emph{л}~}            \\
        \hline
    \end{tabular}
\end{center}
\medskip Сравнивая эти таблицы, видим, что утверждения в правой и левой части принимают одинаковые значения.
\begin{exercise}
    С помощью таблиц истинности докажите формулы \ref{enum:1:1}~--~\ref{enum:1:4}.
\end{exercise}
\par Следующие импликации носят названия:z
\par $T = (A \Rightarrow B)$ ~--~ \emph{прямая теорема},
\par $T = (B \Rightarrow A)$ ~--~ \emph{теорема, обратная к предыдущей}. \\
Справедлива формула:
\begin{center}
    $(A \Rightarrow B) = ((\neg B) \Rightarrow (\neg A))$,
\end{center}
\noindent которая служит основой для распространенного в математике доказательства методом <<от противного>>. Действительно, левая часть формулы ложна тогда и только тогда, когда $A =$ \emph{и}, а $B =$ \emph{л}. Но и правая часть ложна тоже только в этом случае.
\par Следует отметить, что из истинности прямой теоремы еще не следует истинность обратной к ней теоремы, как это видно из примера
%% ! нет примера для ссылки
$1.1$: в разобранном примере из утверждения $B$ не следует утверждение $A$, потому что, как известно, существуют четные числа, не кратные четырем (например, 2).

\newpage
\subsection*{\emph{Предикаты и кванторы}}
\begin{definition}
    Суждение, зависящее от переменной величины, которое при подстанвке значений переменного становится высказыванием, называют \emph{предикатом}.
\end{definition}
\begin{example}
    $A(x) =$ <<студент $x$ учится на физическом факультете>> есть предикат, зависящий от одного переменного $x$. Здесь $A(x)$ -- одноместный предикат. Неравенство $B(x,y) = x^2 + y^2 \geqslant 0$ представляет собой двуместный предикат.
\end{example}
\par\medskip Как и для высказываний, с помощью логических операций $\vee, \wedge, \neg, \Rightarrow, \Leftrightarrow$ можно строить новые предикаты, и мы получим \emph{алгебру предикатов}. Новые предикаты и высказывания можно строить из предикатов также с помощью символов, называемых \emph{кванторами}:
\par $\exists$ ~--~ \emph{квантор существования} (читается: <<существует>>, <<найдется>>),
\par $\forall$ ~--~ \emph{квантор всеобщности} (читается: <<для любого>>, <<для всякого>>, <<для всех>>).
\begin{definition}
    $\forall x : A(x)$ (читается: <<\emph{для всех} $x A(x)$>>) ~--~ высказывание, которое истинно, если предикат $A(x)$ истинен для всех $x$ из его области определение, и ложно ~--~ в противном случае.
\end{definition}
\begin{example}
    Утверждение <<любой студент УрФУ учится на физическом факультете (ФФ)>>, которое формально можно записать в виде:
    \begin{center}
        $\forall x \in$ УрФУ ~:~ \{студент $x$ учится на ФФ\},
    \end{center}
    есть ложное высказывание.
\end{example}
\par Утверждение <<квадрат действительного числа есть число неотрицательное>>, формально записывается в виде:
\begin{center}
    $\forall x \in \mathbb{R} : x^2 \geqslant 0$,
\end{center}
есть истинное высказывание.
\begin{definition}
    $\exists x : A(x)$ (читается: <<существует $x, A(x)$)>> ~--~ высказывание, которое истинно, если предикат $A(x)$ истинен на каком-то конкретном $x$ из его области определения, и ложно, если предикат $A(x)$ ложен при всех $x$ из его области определения.
\end{definition}
\begin{example}
    Утверждение <<на физическом факультете учатся девушки>>, которое формально записывается в виде:
    \begin{center}
        $\exists x :$ \{девушка $x$ учится на физическом факультете\}
    \end{center}
    есть истинное высказывание, так как, конечно, на физическом факультете учится хотя бы одна девушка.
    \par Высказывание $\exists x \in \mathbb{R} : x^2 + 1 \leqslant 0$ ~--~ ложное.
\end{example}
\par Построим отрицание высказывания $\forall x : A(x)$. Если данное утверждение не имеет места, то суждение $A(x)$ имеет место не для всех $x$, т. е. существует элемент $x$, для которого $A(x)$ не имеет места:
\begin{center}
    $\neg (\forall x: A(x)) = (\exists x: \neg A(x))$.
\end{center}
\begin{example}
    Один маленький мальчик очень грамотно построил отрицание утверждения <<Все дети любят кашу>>. Он подумал и, решительно отодвинув от себя тарелку, сказал: <<Не все>>.
\end{example}
\par Совершенно аналогично
\begin{center}
    $\neg (\exists x: A(x)) = (\forall x: \neg A(x))$.
\end{center}
\par Таким образом, чтобы построить отрицание логической формулы, содержащей кванторы, необходимо квантор $\forall$ заменить на квантор $\exists$, а квантор $\exists$ заменить на квантор $\forall$ и предикат заменить на его отрицание.
\begin{example}
    \begin{multline}
        \neg (\forall x: \{\exists y: [\forall z: A(x,y,z)]\}) = \\ = \exists x: \neg \{\exists y: [\forall z: A(x,y,z)]\} = \\ = \exists x: \{\forall y: \neg [\forall z: A(x,y,z)]\} = \\ = \exists x: \{\forall y: [\exists z: \neg A(x,y,z)]\}.
    \end{multline}
\end{example}
\par Заметим, что скобки и двоеточия можно частично или вовсе опускать при записи такого рода логических формул, если не возникает разночтений.

\newpage
\section{~Элементы теории множеств}
\subsection{~Понятие множества}
\par Понятие множества в математике рассматривается как первичное, неопределяемое понятие.
\par Под \emph{множеством} будем понимать совокупность (или семейство, или собрание, или класс) объектов, обладающих определенным признаком. Например, множество деревьев в парке, множество звезд на небе, множество студентов в аудитории, множество натуральных чисел, множество корней уравнения $x^2 - 3x + 2 = 0$, множество треугольников на плоскости и т.д.
\par Объекты, обладающие этим признаком, называются \emph{элементами множества}. Множества будем обозначать заглавными буквами $A, B, C$,..., а элементы этих множеств -- строчными буквами $a, b, c$,.... Множество \emph{содержит} элементы, а элементы \emph{принадлежат} множеству. Для обозначения принадлежности используется знак $\in$. Если $a$ элемент множества $A$, то этот факт записывается так: $a \in A$. Запись $b \notin A$ означает, что элемент $b$ не принадлежит множеству $A$. Так, имеем $3 \in \mathbb{Z}$ и $\frac{1}{2} \notin \mathbb{Z}$.
\par Множества делятся на \emph{конечные} и \emph{бесконечные}. Например, множества $\mathbb{N}, \mathbb{Z}, \mathbb{Q}, \mathbb{R}$ бесконечны, а множество корней уравнения $x^2 -5x + 6 = 0$ конечно. Множество, не содержащее элементов, называется \emph{пустым}. Пустое множество обозначается символов $\emptyset$ Запись
\begin{center}
    $C = \{x \in B~|~P(x)\}$
\end{center}
обозначает множество, состоящее из всех тех и тлько тех элементов множества $B$, которое обладает свойством $P(x)$.
\par Например, $\{x \in \mathbb{R}~|~0 \leqslant x \leqslant 1$ ~--~ множество тех действительных чисел, которые обладают свойством быть не меньше нуля и не больше единицы.
\begin{definition}
    Множество $Y$ называется \emph{подмножеством} множества $X$, если любой элемент множества $Y$ является элементом множества $X$. Это обозначается записью $Y \subseteq X$.
    \par В кванторах это определение можно записать следующим образом:
    \begin{center}
        $\forall y \in Y : y \in X$.
    \end{center}
    \par Например, $\{1,2\} \subseteq \{1,2,-\frac{1}{2}\},~~\mathbb{N} \subseteq \mathbb{Z},~~\mathbb{Z} \subseteq \mathbb{Q},~~\mathbb{Q} \subseteq \mathbb{R}$.
    \par Из определения подмножества следует, что всякое множество является подмножеством самого себя, а пустое множество является подмножеством любого множества.
\end{definition}
\begin{example}
    Решим следующую простую задачу: найти все подмножества множества $\{1,2,3\}$. Очевидно, что это множество имеет восемь подмножеств: $\emptyset, \{1\}, \{2\}, \{3\}, \{1,2\}, \{1,3\}, \{2,3\}$ и $\{1,2,3\}$.
\end{example}
\begin{definition}
    Множество $X$ \emph{равно} множеству $Y$, если любой элемент множества $X$ является элементом множества $Y$ и любой элемент множества $Y$ является элементом множества $X$. Другими словами, если $X$ и $Y$ состоят из одних и тех же элементов. На письме это обозначается обычным образом: $X=Y$.
\end{definition}
\par Из определения следует, что множества $X$ и $Y$ равны тогда и только тогда, когда $X \subseteq Y$ и $Y \subseteq X$.
\begin{example}
    Множество корней уравнения
    \begin{center}
        $x^3 - 6x^2 + 11x - 6 = 0$
    \end{center}
    равно множеству $\{1,2,3\}$, а множество $\{51,2,3\}$ равно множеству $\{3,2,51,2\}$.
\end{example}
\par Заметим, что, перечисляя элементы множества, принято записывать каждый из них только один раз. Подчеркнем, что по определению равенство неважно, в каком порядке перечисляются элементы.

\subsection{~Операции над множествами}
\par Рассмотрим следующие Операции над множествами: пересечение, объединение, дополнение и разность. Эти операции называются \emph{булевыми}. Установим основные свойства булевых операций.
\begin{definition}
    Пусть $X$ и $Y$ ~--~ некоторые множества. Тогда \emph{пересечением} множеств $X$ и $Y$ называется множество
    \begin{center}
        $X \bigcap Y = \{x~|~x \in X$ и $x \in Y\}$,
    \end{center}
    \emph{объединением} множеств $X$ и $Y$ называется множество
    \begin{center}
        $X \bigcup Y = \{x~|~x \in X$ или $x \in Y\}$.
    \end{center}
\end{definition}
\par Например, если $X = \{2,4,6,8,9\},~Y = \{1,3,6,9\}$, то $X \bigcap Y = \{6,9\}$, а $X \bigcup Y = \{1,2,3,4,6,8,9\}$.
\par Отметим основные свойства введенных операций:
\begin{enumerate}[ref=\arabic{enumi}]
    \item $X \bigcap X = X$ ~--~ \emph{идемпотентность пересечения}; \label{enum:2:1}
    \item $X \bigcup X = X$ ~--~ \emph{идемпотентность объединения}; \label{enum:2:2}
    \item $X \bigcap Y = Y \bigcap X$ ~--~ \emph{коммутативность пересечения}; \label{enum:2:3}
    \item $X \bigcup Y = Y \bigcup X$ ~--~ \emph{коммутативность объединения}; \label{enum:2:4}
    \item $(X \bigcap Y) \bigcap Z = X \bigcap (Y \bigcap Z)$ ~--~ \emph{ассоциативность пересечения}; \label{enum:2:5}
    \item $(X \bigcup Y) \bigcup Z = X \bigcup (Y \bigcup Z)$ ~--~ \emph{ассоциативность объединения}; \label{enum:2:6}
    \item $X \bigcap (Y \bigcup Z) = (X \bigcap Y) \bigcup (X \bigcap Z)$ ~--~ \emph{дистрибутивность пересечения относительно объединения}; \label{enum:2:7}
    \item $X \bigcup (Y \bigcap Z) = (X \bigcup Y) \bigcap (X \bigcup Z)$ ~--~ \emph{дистрибутивность объединения относительно пересечения}; \label{enum:2:8}
    \item $X \bigcap \emptyset = \emptyset$; \label{enum:2:9}
    \item $X \bigcup \emptyset = X$. \label{enum:2:10}
\end{enumerate}
\par Доказательство отмеченных свойств продемонстрируем на примере тождества \ref{enum:2:8}. Надо доказать, что любой элемент множества $X \bigcup (Y \bigcap Z)$ принадлежит множеству $(X \bigcup Y) \bigcap (X \bigcup Z)$, и обратно.
\par Пусть $a \in X \bigcup (Y \bigcap Z)$. Тогда $a \in X$ или $a \in Y \bigcap Z$. Если $a \in X$, то $a \in X \bigcup Y$ и $a \in X \bigcup Z$, и поэтому $a$ принадлежит их пересечению, т. е. правой части. Если же $a \in Y \bigcap Z$, то $a \in Y$ и $a \in Z$. Отсюда следует, что $a \in X \bigcup Y$ и $a \in X \bigcup Z$, т. е. $a$ принадлежит и их пересечению.
\par Докажем обратное включение. Пусть теперь $a \in (X \bigcup Y) \bigcap (X \bigcup Z)$. Тогда $a \in X \bigcup Y$ и $a \in X \bigcup Z$. Если $a \in X$, то $a \in X \bigcup (Y \bigcap Z)$. Если же $a \notin X$, то из условий $a \in X \bigcup Y$ и $a \in X \bigcup Z$ следует, что $a \in Y$ и $a \in Z$, а значит, $a \in Y \bigcap Z$, следовательно, $a \in X \bigcup (Y \bigcap Z)$.~~~~$\square$
\end{document}
