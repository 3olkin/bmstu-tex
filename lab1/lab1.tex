\documentclass[a4paper, 12pt]{article}
\usepackage{cmap}
\usepackage{mathtext}
\usepackage{amsmath,amssymb}
\usepackage[T2A]{fontenc}
\usepackage[utf8]{inputenc}
\usepackage[english, russian]{babel}
\usepackage{bmstu-lab}
\usepackage{mathtools}


\begin{document}
\graphicspath{{images/}{images2/}} % папки с картинками

\worknumber{1}
\variant{10}
\workname{Математика в \LaTeX}
\discipline{Автоматизация процессов разработки научно-технической документации}
\group{ИУ6-65Б}
\date{27.02.21}
\author{А.Н.Золкин}
\tutor[Преподаватель]{Т.А.Ким}
\bmstutitlelab

\section{Матрицы}
\subsection{Задание}
$$
  \begin{bmatrix}
    \frac{1}{\infty} & 2      \\
    a                & b      \\
    c                & \aleph
  \end{bmatrix}
$$
\subsection{Дополнительно}
$$
  \begin{pmatrix}
    1      & 2     & 3      \\
    a      & b     & c      \\
    \alpha & \beta & \infty
  \end{pmatrix}
  \hspace{1cm}
  \begin{Bmatrix}
    1         & 2            & 3        \\
    a         & b            & c        \\
    \rho\beta & \sigma\alpha & \delta^2
  \end{Bmatrix}
$$

\section{Интеграл}
\[ \int_{\beta_0C}^{\beta_nA^\theta} f_\beta\left(\theta_\beta\right) \,d\theta_\beta \geqslant \zeta_\beta\]

\section{Сумма}
\[ \frac{ \sum_{i=0}^{N}x_i\alpha\beta + 121}{1 + \Delta(N+C)}\]
\vspace{1cm}
\[ \frac{\displaystyle \sum_{i=0}^{N} \frac{x_i}{1+C} + 23 }{1 + \Delta(N+C)}\]
\section{Два столбика}
\begin{alignat}{3}
   & \frac{ \sum_{i=0}^{N}x_i\alpha\beta + 121}{1 + \Delta(N+C)} &  &  & \int_{\beta_0C}^{\beta_nA^\theta} f_\beta\left(\theta_\beta\right) \,d\theta_\beta \geqslant \zeta_\beta \label{eq1} \\
   & a_1 < a_2, a_2 \text{--- простое}_\text{число}              &  &  & a_2\& a_3 \label{eq2}
\end{alignat}
Ссылки \eqref{eq1}, \eqref{eq2} необходимы для правильной нумерации формул.
\section{Почти то же самое}
\begin{equation} \label{eq3}
  \begin{aligned}
    \begin{aligned}
       & \frac{ \sum_{i=0}^{N}x_i\alpha\beta + 121}{1 + \Delta(N+C)} & \int_{\beta_0C}^{\beta_nA^\theta} f_\beta\left(\theta_\beta\right) \,d\theta_\beta \geqslant \zeta_\beta \\
       & a_1 < a_2, a_2 \text{--- простое}_\text{число}              & a_2\& a_3
    \end{aligned}
  \end{aligned}
\end{equation} \\
На странице \pageref{eq3} есть формулы \eqref{eq3}
\end{document}
