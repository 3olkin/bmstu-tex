\documentclass[a4paper,12pt]{article}
\usepackage[utf8]{inputenc}
\usepackage[english,russian]{babel}
\usepackage{bmstu-lab}

\newcounter{nc}[section]
\renewcommand{\thenc}{Счетчик \Roman{nc}.} %% переопределяем формат вывода значения счетчика
\newenvironment{nc}[1][]{\stepcounter{nc}}{}

\newcounter{simple}[section]
\renewcommand{\thesimple}{\arabic{simple}. }
\newenvironment{simple}[1][]{\stepcounter{simple}}{}

%% переопределяем символ для уровней маркированного списка
\renewcommand{\labelitemi}{$\spadesuit$}
\renewcommand{\labelitemii}{$\clubsuit$}
\renewcommand{\labelitemiii}{$\heartsuit$}
\renewcommand{\labelitemiv}{$\diamondsuit$}

%% переопределяем символ для уровней нумерованный списка
\renewcommand{\labelenumi}{\arabic{enumi}.}
\renewcommand{\labelenumii}{\arabic{enumi}.\arabic{enumii}.}
\renewcommand{\labelenumiii}{\arabic{enumi}.\arabic{enumiii}.}
\renewcommand{\labelenumiv}{\arabic{enumi}.\arabic{enumiv}.}

\newenvironment{nenv}
{
    \begin{center}
    \begin{tabular}{|p{0.9\textwidth}|}
        \hline\\
}
{
    \\\\\hline
    \end{tabular}
    \end{center}
}


\begin{document}
\graphicspath{{images/}{images2/}}
\worknumber{3}
\variant{10}
\workname{Русский язык в \LaTeX часть 2}
\discipline{Автоматизация процессов разработки научно-технической документации}
\group{ИУ6-65Б}
\date{18.04.21}
\author{А.Н.Золкин}
\tutor[Преподаватель]{Т.А.Ким}
% \bmstutitlelab

\section{Счетчики}

\begin{nc}
    \thenc \begin{simple} \thesimple Какой-то текст \end{simple} \\
    \thenc \begin{simple} \thesimple Какой-то текст \end{simple} \\
    \thenc \begin{simple} \thesimple Какой-то текст \end{simple} \\
\end{nc}

\section{Окружения}
\begin{enumerate}
    \item \begin{nenv}
              Новое окружение 1
          \end{nenv}
    \item  \begin{nenv}
              Новое окружение 2
          \end{nenv}
\end{enumerate}

\section{Маркированный список}
\begin{itemize}
    \item Заголовок
          \begin{itemize}
              \item Первый уровень
                    \begin{itemize}
                        \item Второй уровень
                              \begin{itemize}
                                  \item Третий уровень
                              \end{itemize}
                    \end{itemize}
          \end{itemize}
\end{itemize}

\section{Нумерованный список}
\begin{enumerate}
    \item Заголовок 1
          \begin{enumerate}
              \item Первый уровень
              \item Первый уровень
                    \begin{enumerate}
                        \item Второй уровень
                        \item Второй уровень
                              \begin{enumerate}
                                  \item Третий уровень
                                  \item Третий уровень
                              \end{enumerate}
                    \end{enumerate}
          \end{enumerate}
    \item Заголовок 2
          \begin{enumerate}
              \item Первый уровень
              \item Первый уровень
                    \begin{enumerate}
                        \item Второй уровень
                        \item Второй уровень
                              \begin{enumerate}
                                  \item Третий уровень
                                  \item Третий уровень
                              \end{enumerate}
                    \end{enumerate}
          \end{enumerate}
\end{enumerate}
\end{document}
