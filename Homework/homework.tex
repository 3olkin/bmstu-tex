%!TEX TS-program = xelatex
\documentclass[a4paper,14pt]{article}
%%% Работа с русским языком
\usepackage[english,russian]{babel} %% загружает пакет многоязыковой вeрстки
\usepackage{fontspec} %% подготавливает загрузку шрифтов Open Type, True Type и др.
\defaultfontfeatures{Ligatures={TeX},Renderer=Basic} %% свойства шрифтов по умолчанию
\setmainfont[Ligatures={TeX,Historic}]{Times New Roman} %% задаeт основной шрифт документа
\usepackage{indentfirst}
\frenchspacing

\usepackage{mathtext}
\usepackage{mathtools}
\mathtoolsset{showonlyrefs=true}
\usepackage{amsmath,amssymb} % AMS
\usepackage{icomma}

%%% Other packages
\usepackage{extsizes} %% расширяет диапазон доступных кеглей
\usepackage{setspace}
\usepackage{pdfpages}
\usepackage{fancyvrb}
\usepackage{fvextra}
\usepackage{icomma}
\usepackage{geometry}
\geometry{top=20mm}
\geometry{bottom=20mm}
\geometry{left=30mm}
\geometry{right=15mm}

%% переопределяем символ для уровней нумерованного списка
\renewcommand{\labelenumi}{\arabic{enumi})}

%%% Пример
\newcounter{example}[section]
\setcounter{example}{1}
\newenvironment{example}{\refstepcounter{example}\par\medskip\textbf{Пример \thesection.\theexample.}~}{\medskip}
%%% Упражнение
\newenvironment{exercise}{\par\medskip\textbf{Упражнение.}~}{\medskip}
%%% Определение
\newcounter{definition}[section]
\setcounter{definition}{5}
\newenvironment{definition}{\refstepcounter{definition}\par\medskip\textbf{Определение \thesection.\thedefinition.}~}{\medskip}

\begin{document}
\onehalfspacing
%% Заглушка для секции, которая должна быть где-то раньше в книге
\section*{}
\setcounter{section}{1}
\noindent и достаточно $B$>>), которое истинно, когда высказывания $A$ и $B$ истинны или ложны одновременно.
\par Наряду с $A \Leftrightarrow B$ в эквиваленции используется запись: $A = B$.
\par Итак, при помощи логических операций построено множество высказываний, которое называют \emph{алгеброй высказываний}.
\par Основные формулы алгебры высказываний следующие:
\begin{enumerate}
    \item $\neg (\neg A) = A$ \label{enum:1:1};
    \item законы \emph{дистрибутивности} (распределительные законы): \\
          $A \wedge (B \vee C) = (A \wedge B) \vee (A \wedge C)$; \\
          $A \vee (B \wedge C) = (A \vee B) \wedge (A \vee C)$;
    \item законы де Моргана\footnote[2]{\small{~\emph{Огастес де Морган} (1806-1871) ~--~ шотландский математик, логик, основоположник логической теории отношений}}: \\
          $\neg(A \wedge B) = (\neg A) \vee (\neg B)$; \\
          $\neq(A \vee B) = (\neg A) \wedge (\neg B)$;
    \item $A \Rightarrow B = (\neg A) \vee B$ \label{enum:1:4};
    \item $\neg(A \Rightarrow B) = A \wedge (\neg B)$ \label{enum:1:5}.
\end{enumerate}
\par Эти формулы могут быть доказаны сравнением соответствующих таблиц истинности.
\begin{example}
    Докажем формулу \ref{enum:1:5}). Таблица истинности для утверждения в левой части имеет вид:
    \begin{center}
        \begin{tabular}{|c|c cc ccc|}
            \hline
            ~$A$~      & ~$B$~      & \multicolumn{2}{|c|}{~$A \Rightarrow B$~} & \multicolumn{3}{|c|}{~$\neg(A \Rightarrow B)$~} \\
            \hline
            ~\emph{и}~ & ~\emph{и}~ & \multicolumn{2}{|c|}{~\emph{и}~}          & \multicolumn{3}{|c|}{~\emph{л}~}                \\
            \hline
            ~\emph{и}~ & ~\emph{л}~ & \multicolumn{2}{|c|}{~\emph{л}~}          & \multicolumn{3}{|c|}{~\emph{и}~}                \\
            \hline
            ~\emph{л}~ & ~\emph{и}~ & \multicolumn{2}{|c|}{~\emph{и}~}          & \multicolumn{3}{|c|}{~\emph{л}~}                \\
            \hline
            ~\emph{л}~ & ~\emph{л}~ & \multicolumn{2}{|c|}{~\emph{и}~}          & \multicolumn{3}{|c|}{~\emph{л}~}                \\
            \hline
        \end{tabular}
    \end{center}
\end{example}

\newpage
\noindent А таблица истинности для утверждения в правой части имеет вид:
\begin{center}
    \begin{tabular}{|c|c cc ccc|}
        \hline
        ~$A$~      & ~$B$~      & \multicolumn{2}{|c|}{~$\neg B$~} & \multicolumn{3}{|c|}{~$A \wedge (\neg B)$~} \\
        \hline
        ~\emph{и}~ & ~\emph{и}~ & \multicolumn{2}{|c|}{~\emph{л}~} & \multicolumn{3}{|c|}{~\emph{л}~}            \\
        \hline
        ~\emph{и}~ & ~\emph{л}~ & \multicolumn{2}{|c|}{~\emph{и}~} & \multicolumn{3}{|c|}{~\emph{и}~}            \\
        \hline
        ~\emph{л}~ & ~\emph{и}~ & \multicolumn{2}{|c|}{~\emph{л}~} & \multicolumn{3}{|c|}{~\emph{л}~}            \\
        \hline
        ~\emph{л}~ & ~\emph{л}~ & \multicolumn{2}{|c|}{~\emph{и}~} & \multicolumn{3}{|c|}{~\emph{л}~}            \\
        \hline
    \end{tabular}
\end{center}
\medskip Сравнивая эти таблицы, видим, что утверждения в правой и левой части принимают одинаковые значения.
\begin{exercise}
    С помощью таблиц истинности докажите формулы \ref{enum:1:1}~--~\ref{enum:1:4}.
\end{exercise}
\par Следующие импликации носят названия:z
\par $T = (A \Rightarrow B)$ ~--~ \emph{прямая теорема},
\par $T = (B \Rightarrow A)$ ~--~ \emph{теорема, обратная к предыдущей}. \\
Справедлива формула:
\begin{center}
    $(A \Rightarrow B) = ((\neg B) \Rightarrow (\neg A))$,
\end{center}
\noindent которая служит основой для распространенного в математике доказательства методом <<от противного>>. Действительно, левая часть формулы ложна тогда и только тогда, когда $A =$ \emph{и}, а $B =$ \emph{л}. Но и правая часть ложна тоже только в этом случае.
\par Следует отметить, что из истинности прямой теоремы еще не следует истинность обратной к ней теоремы, как это видно из примера
%% ! нет примера для ссылки
$1.1$: в разобранном примере из утверждения $B$ не следует утверждение $A$, потому что, как известно, существуют четные числа, не кратные четырем (например, 2).

\newpage
\subsection*{\emph{Предикаты и кванторы}}
\begin{definition}
    Суждение, зависящее от переменной величины, которое при подстанвке значений переменного становится высказыванием, называют \emph{предикатом}.
\end{definition}
\begin{example}
    $A(x) =$ <<студент $x$ учится на физическом факультете>> есть предикат, зависящий от одного переменного $x$. Здесь $A(x)$ -- одноместный предикат. Неравенство $B(x,y) = x^2 + y^2 \geqslant 0$ представляет собой двуместный предикат.
\end{example}
\par\medskip Как и для высказываний, с помощью логических операций $\vee, \wedge, \neg, \Rightarrow, \Leftrightarrow$ можно строить новые предикаты, и мы получим \emph{алгебру предикатов}. Новые предикаты и высказывания можно строить из предикатов также с помощью символов, называемых \emph{кванторами}:
\par $\exists$ ~--~ \emph{квантор существования} (читается: <<существует>>, <<найдется>>),
\par $\forall$ ~--~ \emph{квантор всеобщности} (читается: <<для любого>>, <<для всякого>>, <<для всех>>).
\begin{definition}
    $\forall x : A(x)$ (читается: <<\emph{для всех} $x A(x)$>>) ~--~ высказывание, которое истинно, если предикат $A(x)$ истинен для всех $x$ из его области определение, и ложно ~--~ в противном случае.
\end{definition}
\begin{example}
    Утверждение <<любой студент УрФУ учится на физическом факультете (ФФ)>>, которое формально можно записать в виде:
    \begin{center}
        $\forall x \in$ УрФУ ~:~ \{студент $x$ учится на ФФ\},
    \end{center}
    есть ложное высказывание.
\end{example}
\par Утверждение <<квадрат действительного числа есть число неотрицательное>>, формально записывается в виде:
\begin{center}
    $\forall x \in \mathbb{R} : x^2 \geqslant 0$,
\end{center}
есть истинное высказывание.
\begin{definition}
    $\exists x : A(x)$ (читается: <<существует $x, A(x)$)>> ~--~ высказывание, которое истинно, если предикат $A(x)$ истинен на каком-то конкретном $x$ из его области определения, и ложно, если предикат $A(x)$ ложен при всех $x$ из его области определения.
\end{definition}
\begin{example}
    Утверждение <<на физическом факультете учатся девушки>>, которое формально записывается в виде:
    \begin{center}
        $\exists x :$ \{девушка $x$ учится на физическом факультете\}
    \end{center}
    есть истинное высказывание, так как, конечно, на физическом факультете учится хотя бы одна девушка.
    \par Высказывание $\exists x \in \mathbb{R} : x^2 + 1 \leqslant 0$ ~--~ ложное.
\end{example}
\par Построим отрицание высказывания $\forall x : A(x)$. Если данное утверждение не имеет места, то суждение $A(x)$ имеет место не для всех $x$, т. е. существует элемент $x$, для которого $A(x)$ не имеет места:
\begin{center}
    $\neg (\forall x: A(x)) = (\exists x: \neg A(x))$.
\end{center}
\begin{example}
    Один маленький мальчик очень грамотно построил отрицание утверждения <<Все дети любят кашу>>. Он подумал и, решительно отодвинув от себя тарелку, сказал: <<Не все>>.
\end{example}
\par Совершенно аналогично
\begin{center}
    $\neg (\exists x: A(x)) = (\forall x: \neg A(x))$.
\end{center}
\par Таким образом, чтобы построить отрицание логической формулы, содержащей кванторы, необходимо квантор $\forall$ заменить на квантор $\exists$, а квантор $\exists$ заменить на квантор $\forall$ и предикат заменить на его отрицание.
\begin{example}
    % \begin{eqnarray}
    %     \neg (\forall x: \{\exists y: [\forall z: A(x,y,z)]\}) = \\ \exists x: \neg \{\exists y: [\forall z: A(x,y,z)]\})
    % \end{eqnarray}
    \begin{multline}
        \neg (\forall x: \{\exists y: [\forall z: A(x,y,z)]\}) = \\\exists x: \neg \{\exists y: [\forall z: A(x,y,z)]\} = \\\exists x: \{\forall y: \neg [\forall z: A(x,y,z)]\} =
        \\\exists x: \{\forall y: [\exists z: \neg A(x,y,z)]\}.
    \end{multline}
\end{example}
\par Заметим, что скобки и двоеточия можно частично или вовсе опускать при записи такого рода логических формул, если не возникает разночтений.
\end{document}
